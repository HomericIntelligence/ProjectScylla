\documentclass[11pt]{article}

% ----------------------------
% Packages
% ----------------------------
\usepackage[margin=1in]{geometry}
\usepackage{times}
\usepackage{graphicx}
\usepackage{amsmath, amssymb}
\usepackage{booktabs}
\usepackage{multirow}
\usepackage{hyperref}
\usepackage{caption}
\usepackage{subcaption}
\usepackage{float}
\usepackage{xcolor}

% ----------------------------
% Title Information
% ----------------------------
\title{<Paper Title>\\
\large <Subtitle (Optional)>}

\author{%
<Author Name 1>\\
<Affiliation>\\
\texttt{<email>}\\[1em]
<Author Name 2>\\
<Affiliation>\\
\texttt{<email>}
}

\date{<Date>}

% ----------------------------
% Document
% ----------------------------
\begin{document}
\maketitle

% ----------------------------
% Abstract
% ----------------------------
\begin{abstract}
<Abstract summarizing motivation, methodology, evaluated models, key results, and conclusions.>
\end{abstract}

\vspace{1em}

\noindent\textbf{Keywords:} <LLM agents; benchmarking; cost-of-pass; multi-agent systems; software engineering>

% ==================================================
\section{Summary}
<High-level executive summary of goals, experimental setup, major findings, and implications.>

% ==================================================
\section{Introduction}
<Problem statement and motivation.>

<Background on LLM agents and agentic architectures.>

<Research questions and hypotheses.>

<Contributions of this paper.>

% ==================================================
\section{Related Work}
<Prior work on LLM benchmarking.>

<Related work on agentic and multi-agent architectures.>

<Gaps addressed by this work.>

% ==================================================
\section{Test Methodology}

\subsection{Experimental Design}
<Overall experimental design and rationale.>

<Tiered evaluation and ablation strategy (T0--T6).>

\subsection{Dimensional Search Space}
<Definition of explored dimensions.>

\begin{itemize}
    \item \textbf{Agent Complexity:} <Tier 0--6 definition>
    \item \textbf{Prompt Complexity:} <Prompt scale 0--10>
    \item \textbf{Skill Complexity:} <Definition and categorization>
    \item \textbf{Agent Hierarchy:} <Flat, hierarchical, hybrid>
\end{itemize}

% ==================================================
\section{Test Metrics}

\subsection{Performance Metrics}
<Completion, success, and accuracy metrics.>

<Fine-grained progress rate definition.>

\subsection{Quality Metrics}
<Implementation rate and semantic validation.>

<Code quality and maintainability metrics.>

\subsection{Efficiency and Cost Metrics}
<Latency measurement.>

<Token usage accounting.>

<Cost-of-Pass (CoP) definition and computation.>

% ==================================================
\section{Test Configuration}

\subsection{Hardware and Infrastructure}
<Compute environment and hardware specifications.>

\subsection{Software Stack}
<Frameworks, libraries, orchestration, and evaluation tooling.>

\subsection{Model Configuration}
<Model versions, context limits, decoding parameters.>

% ==================================================
\section{Test Cases}

\subsection{Pull Request Selection}
<PR selection methodology and constraints.>

\subsubsection{PR Size Categories}
\begin{itemize}
    \item <Small: $<$100 LOC>
    \item <Medium: 300--500 LOC>
    \item <Large: 500--2000 LOC>
\end{itemize}

\subsection{Workflow Categories}
<Description of each workflow category.>

\begin{itemize}
    \item \textbf{Build System:} <Description>
    \item \textbf{CI/CD:} <Description>
    \item \textbf{Bug Fixing:} <Description>
    \item \textbf{New Features:} <Description>
    \item \textbf{Refactoring:} <Description>
    \item \textbf{Optimization:} <Description>
    \item \textbf{Review:} <Description>
    \item \textbf{Documentation:} <Description>
    \item \textbf{Issue Filing:} <Description>
\end{itemize}

\subsection{Test Case Matrix}
<Table mapping PRs, workflows, and complexity tiers.>

% ==================================================
\section{Model Summary}

\subsection{Claude Code}
\begin{itemize}
    \item \textbf{Claude Opus:} <Role and configuration>
    \item \textbf{Claude Sonnet:} <Role and configuration>
    \item \textbf{Claude Haiku:} <Role and configuration>
\end{itemize}

\subsection{OpenAI}
\begin{itemize}
    \item \textbf{Codex / GPT-5.2:} <Role and configuration>
\end{itemize}

\subsection{Large Model CLI-Based Systems}
\begin{itemize}
    \item Claude Opus
    \item OpenAI GPT-5.2
    \item Gemini 3.0 Pro
    \item DeepSeek
    \item Qwen 3
    \item MBZ-K2
    \item Kimi-K2 + Kimi-3
\end{itemize}

<Unified description of CLI-based workflows.>

% ==================================================
\section{Results}

\subsection{Quantitative Results}
<Tables and figures summarizing performance, quality, and cost.>

\subsection{Comparative Analysis}
<Comparison across models, tiers, and workflows.>

\subsection{Cost--Performance Trade-offs}
<Analysis of scaling behavior and diminishing returns.>

% ==================================================
\section{Discussion}
<Interpretation of results.>

<Failure modes and limitations.>

<Implications for agent design.>

% ==================================================
\section{Conclusions}
<Summary of findings.>

<Answers to research questions.>

<Key takeaways.>

% ==================================================
\section{Further Work}
<Future benchmarks, extensions, and research directions.>

% ==================================================
\section*{Acknowledgements}
<Acknowledgements and funding sources.>

% ==================================================
\bibliographystyle{plain}
\bibliography{<bibliography-file>}

% ==================================================
\appendix

\section{Detailed Metric Definitions}
<Expanded formulas and definitions.>

\section{Additional Tables and Figures}
<Supplementary material.>

\section{Reproducibility Checklist}
<Steps and artifacts required to reproduce results.>

\end{document}
